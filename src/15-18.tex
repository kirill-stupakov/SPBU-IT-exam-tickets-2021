\documentclass{article}
\usepackage[T2A]{fontenc} 
\usepackage[utf8]{inputenc}
\usepackage[english,russian]{babel}
\usepackage{amssymb} 
\usepackage{amsmath}
\DeclareMathOperator{\sign}{sign}
\usepackage{indentfirst}

\title{tickets15-18}

\begin{document}
%\AmS-\LaTeX
 

\section{15. Ограниченные сумма и произведение. Примеры: признак делимости, число делителей,
свойство простоты. Разрешимые предикаты и разбор случаев. Алгебра разрешимости.}

\emph{В старых билетах их номера 22-23}\\

\subsection{Примеры}
\emph{Лекция 9}
\subsubsection{Признак делимости} 
    $d: N^{2} \rightarrow N;$ \\
    d(x, y) =
    $\left\{ 
        \begin{aligned} 
        1,\; x \mid y \\
        0,\; x \nmid y
    \end{aligned}
    \right. $
    \\Считаем, что $0 \mid 0$ и $0 \nmid y$ при $y \neq 0$\\ 
    \\Доказательство примитивной рекурсивности:\\
    Смотрим на знак остатка:\\
    d(x, y) = $\overline{sg}(x\; mod\; y)$
    \begin{flushright}
        $\blacksquare$
    \end{flushright}
    
\subsubsection{Число делителей}
    $D: N \rightarrow N;$\\
    D(y) = 
    $\sum\limits_{i \leq y} d(i, y)$
    \\Считаем, что D(0) = 1\\
    \\Доказательство примитивной рекурсивности:\\
    Основано на приведённых ниже теоремах и дано в качестве упражнения
    \begin{flushright}
        $\blacksquare$
    \end{flushright}
    \textbf{Теорема об ограниченных сумме и произведении}\\
    Пусть $\bar x = (x_{1}, ... , x_{n}) \in N^{n},\; y, z \in N,\; f:\;N^{n+1} \rightarrow N$, f -- примитивно-рекурсивная функция ($f \in PRC$)\\
    Тогда:
    \begin{enumerate}
        \item $\prod\limits_{z < y}f(\bar x , z) \in PRC\;\text{, при условии, что }\prod\limits_{z < 0}f(\bar x , z) = 1$
        \item Аналогично для $\sum\limits_{z < y}f(\bar x , z) \in PRC\;\text{, при условии, что }\sum\limits_{z < 0}f(\bar x , z) = 0$
    \end{enumerate}
    Доказательство:\\
    Оба утверждения доказываются аналогично, поэтому ограничимся только первым из них.\\
    По договорённости: $\prod\limits_{z < 0}f(\bar x , z) = 1$, значит, одна базовая функция известна.\\
    $\prod\limits_{z < y+1}f(\bar x , z) = (\prod\limits_{z < y}f(\bar x , z) \cdot f(\bar x, y))$ -- расписали от y+1, а произведение -- тоже примитивно-рекурсивная функция;\\
    Таким образом: '$\prod\limits_{}<$' = PR($\bar 1$, g), где g($\bar x$, y, z) = $z \cdot f(\bar x, y)$
    \begin{flushright}
        $\blacksquare$
    \end{flushright}
    \textbf{Следствие}\\
    Если f($\bar x$, y) и k($\bar x$, w), где $k \in PRC,\; w\in$ N, то:\\
    \begin{itemize}
        \item $\sum\limits_{z < k(\bar x, w)}f(\bar x , z) \in PRC$
        \item $\prod\limits_{z < k(\bar x, w)}f(\bar x , z) \in PRC$
    \end{itemize}
    Доказательство:\\
    Оставлено в качестве упражнения.
    \begin{flushright}
        $\blacksquare$
    \end{flushright}
    
\subsubsection{Признак простоты}\\
    $Pr(x): N \rightarrow N;$\\
    Pr(x) = 
    $\left\{ 
        \begin{aligned} 
        1,\; x - \text{Простое}\;\;\;\;\;\\
        0,\; x - \text{Не простое}
    \end{aligned}
    \right. $
    \\Считаем, что 0 -- простое число\\
    \\Доказательство примитивной рекурсивности:\\
    Pr(x) = $\overline{sg}(|D(x) - 2|$)
    \begin{flushright}
        $\blacksquare$
    \end{flushright}
        
        
     

\subsection{Разрешимые предикаты}
\emph{Лекции 7 и 9}\\

    Предикат называется \textbf{разрешимым}, если его характеристическая функция вычислима. \\
    
    Смысл:
        
        Характеристическая функция вычислима, следовательно, можно построить алгоритм по вычислению истинности или ложности утверждения за конечное время. Значит, вне зависимости от значения аргумента, вычисления \textbf{всегда} закончатся и дадут значение 1 или 0. Таким образом, характеристическая функция предиката -- всюду определённая функция.
    \begin{itemize}
        \item Пример разрешимого предиката:\\
                Предикат (или отношение) x > 0. Характиристическая функция -- знак числа.
        \item Пример неразрешимого предиката:\\
                10-я проблема Гилберта: "Всегда ли имеются корни у какого-либо полинома -- эта проблема неразрешима".
    \end{itemize}
    Выделяют важный подкласс разрешимых предикатов -- таких, у которых характеристическая функция примитивно-рекурсивна.

\subsubsection{Разбор случаев}
    Пусть $\exists$
        
        $f_{1}, ... , f_{k}$ -- набор вычислимых (или, в частном случае, примитивно-рекурсивных) функций
        
        $P_{1}, ... , P_{k}$ -- набор разрешимых (или, в частном случае, примитивно-рекурсивных) предикатов
        
    Пусть также при $\forall$ значении аргумента ровно один из них точно является истинным.\\
    
    Тогда можно определить функцию g(x):
    
    g(x) = 
        $\left\{ 
            \begin{aligned} 
            f_{1},\; \text{если} \;P_{1}(x)\\
            ...\\
            f_{k},\; \text{если} \;P_{k}(x)\\
        \end{aligned}
        \right. $
    
    Такая функция вычислима (либо, если всё до неё было примитивно-рекурсивным, то она примитивно-рекурсивна).\\
    
    
    Доказательство:\\
    Доказательство получается, если положим g(x) как линейную комбинацию функций, умноженных на знак соответствующего предиката (0 или 1):\\
    g(x) = $f_{1}(x)c_{P_{1}} + ... + f_{1}(k)c_{P_{k}}$
    \begin{flushright}
        $\blacksquare$
    \end{flushright}


\subsection{Алгебра разрешимости}
\emph{Лекция 9}\\

    Определим алгебру разрешимости, в которой можно выполнять приведённые ниже логические операции\\
    Если P, Q -- разрешимы (примитивно-рекурсивны), тогда разрешимы (примитивно-рекурсивны) и следующие предикаты
    \begin{itemize}
        \item $\bar P$ ("не P");\\
        Доказательство:
        \\ Строим функцию, значение которой равно 1 - P (или 1 - характеристическая функция P);
        \begin{flushright}
            $\blacksquare$
        \end{flushright}
        \item $P \wedge Q$;\\
        Доказательство:
        \\ Строим функцию, значение которой равно обратному знаку умножения P и Q;
        \begin{flushright}
            $\blacksquare$
        \end{flushright}

        \item $P \vee Q$;\\
        Доказательство:
        \\ Строим функцию, значение которой равно знаку сложения P и Q;
        \begin{flushright}
            $\blacksquare$
        \end{flushright}
    \end{itemize}
    

\section{16. Ограниченная минимизация. Теоремы о ее вычислимости и примитивной рекурсивности.
Ограниченная минимизация и подстановка. Примеры: простые числа и показатель их
степени.}    

\emph{В старых билетах это номер 24}\\

    \subsection{Ограниченная минимизация}
    \emph{Лекция 9}
        
        \subsubsection{Определение 1}
            Пусть $P:\; N^{n+1}\; \rightarrow\; N$ -- предикат, тогда говорят, что функция g: $N^{n+1}\rightarrow N:\\ \forall \bar x \in N^{n+1}, \; \forall y \in N \\$
            g(x) = 
                    $
                    \left\{ 
                        \begin{aligned} 
                            min\; z < y:\; P(\bar x, y)\text{, если такой z } \exists  \\
                            y \text{, иначе}\;\;\;\;\;\;\;\;\;\;\;\;\;\;\;\;\;\;\;\;\;\;\;\;\;\;\;\;\;\;\;\;\;\;\;\;\;\;\;\;\;\;\;\;\\
                        \end{aligned}
                    \right . \\
                    $
            Получена из P с помощью оператора \textbf{ограниченной минимизации}.\\
            
            Обозначение:\\
            g($\bar x, y$) = $\mu_{z < y} P(\bar x, y) = \mu z < y (P(\bar x, y))$\\
            
            Замечание:\\
            Под P можно подразумевать не предикат, а его характеристическую функцию. \\
        
        
        \subsubsection{Определение 2}
        \emph{Частный случай или другая форма}\\
            
            Пусть $f:\; N^{n+1}\; \rightarrow\; N$, тогда говорят, что функция g: $N^{n+1}\rightarrow N:\\ \forall \bar x \in N^{n+1}, \; \forall y \in N \\$
            g(x) = 
                    $
                    \left\{ 
                        \begin{aligned} 
                            min\; z < y:\; f(\bar x, y) = 0\text{, если такой z } \exists  \\
                            y \text{, иначе}\;\;\;\;\;\;\;\;\;\;\;\;\;\;\;\;\;\;\;\;\;\;\;\;\;\;\;\;\;\;\;\;\;\;\;\;\;\;\;\;\;\;\;\;\;\;\;\;\;\;\\
                        \end{aligned}
                    \right . \\
                    $
            Получена из f с помощью оператора \textbf{ограниченной минимизации}.\\
            
            Обозначение:\\
            g($\bar x, y$) = $\bar \mu_{z < y} f(\bar x, y)$\\
            
        \subsubsection{Замечание о связи определений}
        $\bar \mu_{z < y} f(\bar x, y) = \mu_{z < y} (f(\bar x, y) = 0)$
    
    \subsection{Теорема об ограниченной минимизации}
    \emph{Лекция 9}\\
        
        Оператор ограниченной минимизации \textbf{примитивно рекурсивен}.\\
        
        Пусть $f:\; N^{n+1}\; \rightarrow\; N$ и $f \in PRC$ или f -- всюду определённая (иногда такие функции называют тотальными) вычислимая функция, тогда:\\
        
        g($\bar x, y$) = $\bar \mu_{z < y} f(\bar x, y) \in PRC$ или тотальная вычислимая функция соответственно.\\
        
        Доказательство:\\
        $\bar \mu_{z < y} f(\bar x, y)$ может быть вычислена следующим образом:\\
        
        $\sum\limits_{u < y}(\prod\limits_{z \leq u}sg(f(\bar x , z)))$\\
        "Таким образом просуммируются только те единицы, которые получились в результате таких произведений, что в них ни разу f, равное нулю, не встретилось."
        \begin{flushright}
            $\blacksquare$\\
        \end{flushright}
        
        \emph{Другой вариант $\mu_{z < y}$ -- в качестве упражнения}
    
    \subsection{Утверждение}
    \emph{Лекция 9}\\
        Пусть f и k $\in PRC$ или тотальные вычислимые функции, тогда:\\
        такой же является и функция $\mu_{z < k(\bar x, w)}(f(\bar x, z) = 0)$
        
        Доказательство:\\
        Дано в качестве упражнения. (Подстановка k($\bar x$, w) вместо y в вычислимую функцию $\bar \mu_{z < y} (f(\bar x, y) = 0)$)
        \begin{flushright}
            $\blacksquare$\\
        \end{flushright}
        
    \subsection{Примеры}
    \emph{Лекции 9 и 10}\\
        \begin{itemize}
            \item $P:\; N \rightarrow N$, P(x) -- простое число с номером x ("x-е простое число"), при этом считаем, что P(0) = 0.\\
            Таким образом: P(1) = 2, P(2) = 3 и т. д.\\
            
            Доказательство примитивной рекурсивности:\\
            P(0) = 0\\
            P(x+1) = $\mu_{z < P(x)!\; + 2}((z > P(x))\; \&\& $  (z -- простое число))\\
            Это правильно, т.к. $\exists$ утверждение о том, что $\forall n $ на промежутке (n, n! + 1] всегда есть простое число, большее n.
            \begin{flushright}
                $\blacksquare$\\
            \end{flushright}
            
            \item $h:\; N^{2} \rightarrow N$, h(x, y) = $exp_y x$ -- показатель степени P(y) в разложении x на простые множители. \\
            Таким образом, $exp_3 300$ = 2.
            
            Доказательство примитивной рекурсивности:\\
            h(x, y) = $\mu_{z < x} P^{z+1}(y) \nmid x $
            \begin{flushright}
                $\blacksquare$\\
            \end{flushright}
        \end{itemize}
        
        
\section{17. Числа Фибоначчи, примитивная рекурсия, лемма, возвратная рекурсия.}
\emph{В старых билетах это номер 25}\\

    \subsection{Числа Фибоначчи}
    \emph{Лекция 10}\\
    
        $F: N \rightarrow N,\; F(0) = 1, F(1) = 2 \text{ и } \forall n \geq 0\; F(n+2) = F(n+1) + F(n)$, тогда:
        
        $F \in PRC$\\
        \subsubsection{Лемма}
        $\forall n \textgreater 0\; F(n) = \sum_{k = 0}^{[(n + 1)/2]} C_{n-k+1}^{k}$, где [...] -- целая часть числа\\
        \\Доказательство:\\ 
        Рассмотрим множество последовательностей из нулей и единиц длины n, в которых нет двух рядом стоящих единиц. Пусть число таких последовательностей длины n есть А(n).
        \begin{enumerate}
            \item По индукции покажем, что для заданного n число таких последовательностей есть F(n), т.е. А(n) = F(n), $\forall n$.\\
            А(0) = 1, так как существует только одна такая (пустая) последовательность.\\
            А(1) = 2, так как существует две такие последовательности -- '0' и '1'. 
            
            Пусть для $k \leq n\; A(n) = F(n)$. Число последовательностей длины n, у которых на n-ом месте находится нуль, равно А(n - 1) = F(n - 1).\\
            Все последовательности длины n + 1 могут быть построены из последовательностей длины n:
            \begin{itemize}
                \item приписыванием к каждой из них на (n + 1)-е место нуля (= A(n))
                \item тем из них, которые на n месте имеют ноль можно также приписать единицу (= A(n - 1)).
            \end{itemize}
            
            А(n + 1) = A(n) + A(n - 1) =F(n) + F(n − 1) = F(n + 1), тогда A(n) = F(n) 
            
            \item C другой стороны А(n) можно вычислить другим способом.\\
            Заметим, каждая такая (правильная) последовательность длины n может содержать не более [(n+1)/2] единиц. Подсчитаем, сколько существует разных последовательностей длины n, содержащих k единиц, где $0 \leq k \leq [(n + 1)/2]$. \\
            Рассмотрим последовательность из n − k нулей. В этой последовательности имеется n − k + 1 мест для расстановки k единиц. Тогда общее число требуемых (правильных) последовательностей длины n, содержащих k единиц, равно числу вариантов выбора таких мест:
            
            $A(n) = \sum_{k = 0}^{[(n + 1)/2]} C_{n-k+1}^{k}$
        \end{enumerate}
        \begin{flushright}
            $\blacksquare$\\
        \end{flushright}
        
        Доказательство теоремы:\\
        $\forall n$ F(n) = A(n) = значению примитивно-рекурсивной функции.
        \begin{flushright}
            $\blacksquare$\\
        \end{flushright}
        
        
    \subsection{Возвратная рекурсия}
    \emph{Лекция 10}\\
    
        Это ещё один способ порождения новых вычислимых функций. Решение задачи о числах Фибоначчи может быть получено на основе понятия возвратной рекурсии.\\
        
        Определение:\\
        Пусть даны всюду определённые функции f, g и $\alpha_{i}$:\\
        $f: N^{n} \rightarrow N,\; g: N^{n+s+1} \rightarrow N,\; s \geq 1$;\\
        $\alpha_1, ... , \alpha_s: N \rightarrow N:\; \forall y \in N \; \alpha_i (y + 1) \leq y, 1 \leq i \leq s$, тогда\\
        функцию $h: N^{n+1} \rightarrow N:$\\
        \begin{itemize}
            \item $h(\bar x, 0) = f(\bar x)$,
            \item $h(\bar x, y + 1) = g(\bar x, y, h(\bar x, \alpha_1(y + 1)), ... , h(\bar x, \alpha_s(y + 1)))$,
        \end{itemize}
        называют функцией, полученной из $f, g, \alpha_1,\; ...,\; \alpha_s$ с помощью \textbf{возвратной рекурсии}.
        
        \subsubsection{Теорема}
            Если функция h получена из $f, g, \alpha_1,\; ...,\; \alpha_s$ с помощью возвратной рекурсии, то h -- примитивно-рекурсивна \textbf{относительно этих функций} (т.е. h может быть получена из этих функций и базисных примитивно-рекурсивных функций композицией и примитивной рекурсией, так что если исходные функции $\in PRC$, то и h $\in PRC$, если нет -- "приведём определение новым функциям над какими-то заданными, это уже будет частный случай").
            
            Иными словами -- возвратная рекурсия \textbf{не} расширяет (не выводит из) класс примитивно-рекурсивных функций. \\
            \\Доказательство:\\
            Введём вспомогательную функцию $H(\bar x, y) = \prod_{i=0}^y P^{h(\bar x, i)} (i + 1)$,  где функция P(i + 1) -- простое число номера (i + 1)\\
            \\$\forall i \leq  y\; h(\bar x, i) = exp_{i + 1} H(\bar x, y).\;\;\;(*)\\$
            \\Заметим: $h(\bar x, \alpha_j(y + 1)) = exp_{\alpha_j (y + 1)\; +\; 1} H(\bar x, y)$\\
            \\Пытаемся задать примитивно-рекурсивное определение:\\
            \begin{itemize}
                \item $H(\bar x, 0):\\
                H(\bar x, 0) = P^{f(\bar x)}(1).$
                \item $H(\bar x, y + 1):\\
                \text{Распишем}: H(\bar x, y + 1)= H(\bar x, y) \cdot P^{h(\bar x, y + 1)}((y + 1) + 1) \text{, а так как }$\\
                \\$h(\bar x, y + 1) = g(\bar x, y, h(\bar x, \alpha_1 (y+1)),\; ... \; h(\bar x, \alpha_s (y+1))) \text{, тогда }$\\
                \\$H(\bar x, y + 1) = H(\bar x, y) \cdot P^{g(\bar x, y, h(\bar x, \alpha_1 (y+1)),\; ... \; h(\bar x, \alpha_s (y+1)))} ((y + 1) + 1)$
            \end{itemize}
            H -- кандидат на примитивно-рекурсивное определение.\\
            \\Отдельно распишем вспомогательные функции:
            \begin{itemize}
                \item $F(\bar x) = P^{f(\bar x)} (1)$
                \item $G(\bar x, y, z) = z \cdot P^{g(\bar x, y, exp_{\alpha_1 (y + 1) + 1} \cdot z, \; ..., \; exp_{\alpha_s (y + 1) + 1} \cdot z} ((y + 1) + 1),$
            \end{itemize}
            \\Тогда:
            \begin{itemize}
                \item $H(\bar x, 0) = F(\bar x),$
                \item $H(\bar x, y + 1) = G(\bar x, y, H(\bar x, y)).$
            \end{itemize}
            \\Следовательно, H -- примитивно рекурсивна от F, G, H: \\
            \\$H \in PR(F, G) & F, G \in PR(f, g, \alpha_1, \; ..., \; \alpha_s) \Rightarrow H \in PR(f, g, \alpha_1, \; ..., \; \alpha_s).$\\
            \\Учитывая (*), восстанавливаем h:\\
            \\$h(\bar x, y) = exp_{y + 1} H(\bar x, y),$\\
            \\и получаем, что h -- также примитивно-рекурсивна:\\
            \\$h \in PR(f, g, \alpha_1, \; ..., \; \alpha_s)$\\
            \begin{flushright}
                $\blacksquare$\\
            \end{flushright}
            
        \subsubsection{Следствие}
            Последовательность Фибоначчи может быть определена с помощью возвратной рекурсии.\\
            \\Доказательство:\\
            Количество параметров $\bar x = |\bar x|$ = 0, s = 2 -- количество вспомогательных функций.\\
            \begin{itemize}
                \item F(0) = 1
                \item $F(y + 1) = F(\alpha_1 (y + 1)) + F(\alpha_2 (y + 1))$, где \\
                \\$\alpha_1(y) = y - 1,\; \alpha_2(y) = y - 2$
            \end{itemize} 
            \begin{flushright}
                $\blacksquare$\\
            \end{flushright}
            
        
\section{18. Неограниченная минимизация. Теорема о вычислимости минимизации. Примеры
неограниченной минимизации и ее свойства. Обращение функций.}     
\emph{В старых билетах это номер 26}\\   

    \subsection{Неограниченная минимизация}
    \emph{Лекция 10}\\
    
        Определение:\\
        Пусть $f: N^{n + 1} \rightarrow N$, тогда говорят, что функция $h: N^n \rightarrow N$ получена из f с помощью оператора \textbf{неограниченной минимизации}, если
        $\forall \bar x \in N^n\\ h(\bar x) = \mu_y(f(\bar x, y) = 0) = $\\
        $
        \left\{ 
            \begin{aligned} 
                min\; y: \;f(\bar x, y) = 0 \text{ при этом } \forall u < y f(\bar x, y) \text{ -- определена и сходится} \\
                \text{не определена}\;\;\;\;\;\;\;\;\;\;\;\;\;\;\;\;\;\;\;\;\;\;\;\;\;\;\;\;\;\;\;\;\;\;\;\;\;\;\;\;\;\;\;\;\;\;\;\;\;\;\;\;\;\;\;\;\;\;\;\;\;\;\;\;\;\;\;\;\;\;\;\;\;\;\;\;\;\;\;\;\;\;\;\;\;\;\;\\
            \end{aligned}
        \right . \\
        $
        
        \\Замечание 1:\\
        Когда говорят просто 'минимизация', подразумевается неограниченная минимизация. При ограниченной минимизации всегда говорится строго 'ограниченная минимизация'.\\
        
        \\Замечание 2:\\
        Возможны и другие формы определения, например:\\
        $\forall \bar x \in N^n\\ h(\bar x) = \mu_y(f(\bar x, y) = 1) = $\\
        $
        \left\{ 
            \begin{aligned} 
                min\; y: \;f(\bar x, y) = 1 \text{ при этом } \forall u < y f(\bar x, y) \text{ -- определена и сходится} \\
                \text{не определена}\;\;\;\;\;\;\;\;\;\;\;\;\;\;\;\;\;\;\;\;\;\;\;\;\;\;\;\;\;\;\;\;\;\;\;\;\;\;\;\;\;\;\;\;\;\;\;\;\;\;\;\;\;\;\;\;\;\;\;\;\;\;\;\;\;\;\;\;\;\;\;\;\;\;\;\;\;\;\;\;\;\;\;\;\;\;\;\\
            \end{aligned}
        \right . \\
        $
        
        \\Замечание 3:\\
        Оператор минимизации часто называется оператором поиска.
        
        Это важно, т.к. в дальнейшем вместо 1 или 0 можно вписывать какие-либо предикаты, свойства которых могут быть верны или нет
    
    \subsection{Теорема о вычислимости минимизации}
    \emph{Лекция 10}\\
        
        Пусть $f: N^{n + 1} \rightarrow N$ вычислима, тогда функция h, полученная из f с помощью оператора минимизации, также вычислима. (Т.н. замкнутость класса вычислимых функций относительно оператора минимизации)\\
        \\Доказательство:\\
        По тезису Чёрча:\\
        $f(\bar x, 0), f(\bar x, 1), ... $ -- вычисляем. При вычислении есть два случая:
        \begin{itemize}
            \item функция может быть не определена
            \item можно не найти значение
        \end{itemize}
        
        Обе эти проблемы не мешают процессу доказательства, т.к. сам регулярный процесс может быть определён. В целом похоже на доказательство вычислимости примитивно-рекурсивных функций.\\
        \\m = max\{n + 1, r\}, где\\
        n = $|x|$ -- количество параметров и r = максимальный номер регистра ("фактически, регистров всего там будет r + 1"), использованного в программе для вычисления f.\\
        \\Планируем и распределяем регистры:\\
        $\underbrace{\underbrace{R_0, \; ... , \; R_{n - 1}}_{\bar x} , \underbrace{R_n}_{y}}_{\text{параметры для вычисл. f}}, \underbrace{R_{n + 1}, \; ... , \; R_m}_{\text{рабочие регистры для f}}, \underbrace{\underbrace{R_{m + n + 1}}_{\text{копия } \bar x}, \underbrace{R_{m + n + 1}}_{\text{копия y}}, \underbrace{R_{m + n + 2}}_{\text{счётчик}}}_{\text{для модификации исходной прогр. для f}},$\\
        \\$\underbrace{...}_{\text{доп. регистры}}, ...  $\\
        \\Тогда программа минимизации должна просто перевычислять $f(\bar x, 0), \; f(\bar x, 1)$, ... , сохраняя в нужном месте копии, вовремя подчищая регистры, восстанавливая аргументы, прибавляя счётчик и заново запуская процесс, пока не сойдётся -- точно такой же процесс, как с примитивно-рекурсивными функциями. Объяснить подробнее -- упражнение.\\
        \\Тогда по тезису Чёрча делаем вывод, что новая функция является вычислимой.
        \begin{flushright}
            $\blacksquare$\\
        \end{flushright}
        
        \subsubsection{Следствие}
        \emph{Лекция 10}\\
        
            Пусть P($\bar x$, y) -- разрешимый предикат, тогда функция h -- вычислима, где h:\\
            
            $\forall \bar x \in N^n \; \; h(\bar x) = \mu_y(P(\bar x, y))$ = \\
            
            $
            \left\{ 
                \begin{aligned} 
                    min\; y: \;P(\bar x, y) \text{ -- истинно, если такой } y\; \exists\\
                    \text{не определена}\;\;\;\;\;\;\;\;\;\;\;\;\;\;\;\;\;\;\;\;\;\;\;\;\;\;\;\;\;\;\;\;\;\;\;\;\;\;\;\;\;\;\;\;\;\\
                \end{aligned}
            \right . \\
            $
        \\Доказательство:\\
        Оставлено в качестве упражнения.
        \begin{flushright}
            $\blacksquare$\\
        \end{flushright}
        \\Замечание 1:\\
        Часто используемый приём определения вычислимой функции, но нужно помнить, что P -- вычислима\\
        \\Замечание 2:\\
        Если P -- не вычислимый предикат (неформально определён), то эта надстройка будет не над вычислимой функцией\\
        \\Пояснение:\\
        Возможно, что предикат, представленный в следствии, будет не всюду определён. Это не противоречие, а частный случай определения оператора ограниченной минимизации.
        
        
    \subsubsection{Следствие (вывод)}
    $\mu$-оператор выводит из класса примитивно-рекурсивных функций, подтверждая тем самым, что класс вычислимых функций шире класса примитивно-рекурсивный функций.
    
    
    \subsection{Примеры}
    \emph{Лекция 10}\\
        \begin{itemize}
            \item h(x) = $\mu_y (|x - y| = 0)$ -- данная функция всюду определена.
            \item c = h() = $\mu_y(y^2 + 4y + 4 = 0)$ -- при y $\textgreater 0$ не всюду определена, при y $\textless 0$ всюду определена
            \item $h(x) = \mu_y (|y^2 - x| = 0)$ -- корень. Пример показывает, что оператор минимизации может порождать нетотальные функции из тотальных
        \end{itemize}
    
    \subsection{Обращение функций}
    \emph{Лекция 10}\\
        
        Определение:\\
        Пусть $f$ -- одноместная вычислимая функция.\\
        Функция $f^{-1}$ называется обращением функции $f$ (обратный), если:\\
        $\forall x \; f^{-1} = \mu_y (f(y) = x)$\\
        
        Замечание:\\
        Допускается, что $f^{-1}$ может быть не определена.
        
        \subsubsection{Примеры}
            \begin{itemize}
                \item $sg^{-1}$ = 
                 $
                \left\{ 
                    \begin{aligned} 
                        x,\; x = 0 \text{ или } x = 1 \;\;\;\;\;\;\;\;\;\;\\\
                        \text{не определена при } x \textgreater 1\\
                    \end{aligned}
                \right . \\
                $
                \item s(x) = x + 1, тогда\\
                $s^{-1}$ = ? -- в качестве упражнения.
            \end{itemize}
            
        \subsubsection{Утверждение}
        Оператор обращения из вычислимых функций порождает вычислимые.
        
        \\Доказательство:\\
        Оставлено в качестве упражнения (Подсказка: это верно, т.к. "мы просто явно выписали способ порождения обратной функции")
    
        
\end{document}
